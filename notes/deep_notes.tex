\documentclass[a4paper,11pt]{article}
%\documentclass[showpacs,showkeys,aps,prd,nofootinbib,floatfix,amsmath,amssymb]{revtex4}
%\usepackage{jheppub}
\usepackage{amssymb}
\usepackage{amsmath}
\usepackage{amsfonts}
%\usepackage{epigraph}
\usepackage{mathrsfs}
\usepackage{graphics}
\usepackage{graphicx}
\usepackage{bm}
\usepackage{bbm}
\usepackage{dsfont}% Math Blackboard
%\usepackage{tensor} 	% Tensor indices
%\usepackage{slashed}	% Dirac Slash Notation
%\usepackage{tikz,ifthen}% TikZ!
%\usepackage[toc,page]{appendix}
\usepackage{braket}



\newcommand{\dd}{\text{d}}
\newcommand{\bJ}{\mathbf{J}}
\newcommand{\bJt}{\mathbf{J}^{(0)}}
\newcommand{\bT}{\mathbf{T}}
\newcommand{\bI}{\mathbf{I}}
\newcommand{\bd}{\mathbf{d}}
\newcommand{\mc}{\mathcal{M}}
\newcommand{\bj}{\bar{j}}
\newcommand{\pola}{\varepsilon}
\newcommand{\ldot}{\!\cdot\!}


\newcommand{\na}[2]{a_{#1}^{[#2]}}
\newcommand{\nb}[2]{b_{#1}^{[#2]}}
\newcommand{\nz}[2]{z_{#1}^{[#2]}}
\newcommand{\nt}[3]{\theta_{#1 #2}^{[#3]}}
\newcommand{\sg}{ \sigma_{\text{sigmoid}}}

\newcommand{\ff}[2]{#1 \left(#2 \right)}


\newcommand{\blue}[1]{\textcolor{blue}{#1}}
%\newcommand{\sla}[1]{\slashed{#1}}


%\textwidth = 460pt
%\textheight = 560pt
%\oddsidemargin =17pt

\title{Feedforward Neural Network}
%\author{R. \'Angeles-Mart\'inez }
%
%
%\emailAdd{reneang17@gmail.com}


%\abstract{All orders loop evolution and its large $N$ limit.}
%% \keywords{Renormalization, electromagnetic properties.}
%% \pacs{11.10.Gh,11.30.Cp,13.40.Em}
%%\tableofcontents
%
%
\begin{document}



In the lectures notes, one has $\theta^{[l]}_{\text{here}}=\theta^{[l-1]}_{\text{there}}$, where
there refers to the machine learning learning course by Andrew.
We do this change to make the connexion with the 
deep learning specialization lectures.


\section{Structure of dense layers}

%In the lectures notes, one has 
%$\theta^{[l]}_{\text{here}}=\theta^{[l-1]}_{\text{there}}$.
%We do this change to make the connexion with the 
%deep learning specialization lectures.

\begin{equation}
    \na{\mu}{0} \equiv \delta_{\mu0}+\delta_{\mu i} x_i %\label{eq:}
\end{equation}

\begin{equation}
    \na{\mu}{l} \equiv \delta_{\mu0} +\delta_{\mu i }^{[l]}\,\,
    \ff{g^{[l]}}{  \nt{i}{\nu}{l}\na{\nu }{l-1}  }  %\label{eq:}
\end{equation}

\begin{equation}
    \nz{i}{l}\equiv \nt{i}{\nu}{l}\na{\nu }{l-1} %\label{eq:}
\end{equation}

It should be understood that in $\na{i}{l}$ 
one has  $i \in \{1,..., s_l\} $, 
and that in $\na{\mu}{l}$ one has $ \mu \in \{0,..., s_l\}$. The binary cross entropy cross function is

\begin{equation}
    cost=\frac1m \sum^m_n 
    J \Big\vert_{x=x_n,y=y_n}  %\label{eq:}
\end{equation}
where

\begin{equation}
    J\equiv -\sum^{s_L}_k
    \left(
        y_k\ff{\log}{\na{k}{L}}  %\label{eq:}
        +(1-y_k)\ff{\log}{1-\na{k}{L}} 
    \right)
\end{equation}





\section{Backward propagation.  Recursion relation}



%
%To implement gradient descent we need 
%the next partial derivatives ($l\in\{1,...,L\} $)
%
%\begin{equation}
%    \frac{\partial J }{\partial  \nt{j_l}{\mu}{l}  } 
%   =\frac{\partial J  }{\partial \na{j_L}{L} } \frac{\partial \na{j_L}{L} }{\partial \nz{j_L}{L} }
%    \left(
%    \frac{\partial \nz{j_L}{L}  }{\partial \na{j_{L-1}}{L-1}  }
%    \cdots
%    \frac{\partial \na{j_{l+1}}{l+1}   }{\partial \na{j_l}{l}   }
%    \right)
%    \frac{\partial \na{j_l}{l}   }{\partial \nz{j_l}{l}   }
%    \frac{\partial \nz{j_l}{l} }
%        {\partial \nt{j_l}{\mu}{l}  }
%    %\label{eq:}
%\end{equation}

The following recursion relations do the job
\begin{equation}
    \underbrace{
    \frac{\partial J }{\partial  \nt{j_l}{j_{l-1}}{l}  }
    }_{\dd  \nt{j_l}{j_{l-1}}{l} } 
   =
   \underbrace{
    \frac{\partial J  }{\partial \nz{j_l}{l} }
    }_{\dd  \nz{j_{l}}{l} }
    \underbrace{
    \frac{\partial \nz{j_l}{l} }{\partial \nt{j_l}{j_{l-1}}{l}  }
    }_{ \na{j_{l-1}}{l-1} }
    %\label{eq:}
\end{equation}



\begin{equation}
\underbrace{
\frac{\partial J  }{\partial \nz{j_l}{l}   }
}_{\dd  \nz{j_{l}}{l} }
	=
	\underbrace{ 
	\frac{\partial J  }{\partial \nz{j_{l+1}}{l+1}   } 
	\frac{\partial \nz{j_{l+1}}{l+1}   }{ \partial \na{j_{l}}{l}  }
	}_{\dd a_{j_l}^{[l]} }
	\frac{\partial \na{j_{l}}{l}    }{ \partial \nz{j_{l}}{l}  }
	=
	\underbrace{
	\frac{\partial J  }{\partial \nz{j_{l+1}}{l+1}   } 
	 \nt{j_{l+1}}{j_l}{l+1}
	 }_{ \dd a_{j_l}^{[l]}=(\theta^{[l+1]T})_{j_l j_{l+1}} \dd z_{j_{l+1}}^{[l+1]} }
	 g^{[l]'}\left( z_{j_{l}}^{[l]}   \right)
\end{equation}

Hence, in an abbreviated notation one has:
\begin{eqnarray}
	&\dd  \nt{j_l}{j_{l-1}}{l} = \dd  \nz{j_{l}}{l}  \na{j_{l-1}}{l-1}\\
	&\dd a_{j_{l-1}}^{[l-1]}=(\theta^{[l]T})_{j_{l-1} j_{l}} \dd z_{j_{l}}^{[l]}\\
	&\dd  \nz{j_{l}}{l}= \dd a_{j_l}^{[l]} \,\, g^{[l]'}\left( z_{j_{l}}^{[l]}   \right)
\end{eqnarray}

Hence, by providing $\{ \dd a^{[l]}  , \theta^{[l]}\}$ one gets $\{ \dd  \nt{j_l}{\mu}{l}, \dd a_{j_{l-1}}^{[l-1]}\}$ and this process can be iterated. 
The initial condition is


\begin{eqnarray}
    &\dd \theta^{[L]}_{j_L \mu}= \frac{\partial J }{\partial  \nt{j_L}{\mu}{L}  } =
    \underbrace{
    \frac{(\na{j_L}{L} -y_{j_L})}{\na{j_L}{L} (1-\na{j_L}{L-1} )} 
    }_{\dd a_{j_{L}}^{[L]} }
    \,\, g^{[L]'} \left( z_{j_{L}}^{[L]} \right)
   \, \na{\mu}{L-1} \\
   & \dd z^{[L]}_{j_L }= \dd a_{j_{L}}^{[L]}\,
   g^{[L]'} \left( z_{j_{L}}^{[L]} \right)\\
   &\dd \theta^{[L]}_{j_L \mu}=  \dd \nz{j_L }{L}\, \na{\mu}{L-1} 
\end{eqnarray}



This layer actually simplifies because one chooses $g^{[L]}= \sg$, 
which means that $\sg'= \sg (1- \sg) $.


\section{Forward propagation convolutional neural networks}

No padding case.
%In the lectures notes, one has 
%$\theta^{[l]}_{\text{here}}=\theta^{[l-1]}_{\text{there}}$.
%We do this change to make the connexion with the 
%deep learning specialization lectures.

\begin{equation}
    \na{ij}{0} \equiv x_{ij} %\label{eq:}
\end{equation}

\begin{equation}
\nz{ij  }{l} = 
    \nt{r}{s}{l} 
    \na{(i S^{[l]}+r)(j S^{[l]}+s) }{l-1} 
    +b^{[l]}\delta_{ij} 
\end{equation}

\begin{equation}
    \na{ij}{l} \equiv \ff{g^{[l]}}{  
    \nz{ij  }{l}   }  %\label{eq:}
\end{equation}



It should be understood that incides of the activation
of the $l$ layer, $\na{ij}{l}$, run over 
$i\in\{ 0,..., n_{H}^{[l]}-1  \}$ and $j\in\{ 0,..., n_{W}^{[l]} -1 \}$, where
$n_{X}^{[l]}= (n_{X}^{[l-1]}-f^{[l]})/S^{[l]}+1$. Finally, the indices of the 
weights of the $l$ layer, $\nt{r}{s}{l}$ , runs over $r,s\in \{0,..., f^{[l]}-1\}$.


\section{Back prop}

The following recursion relations do the job
\begin{equation}
    \underbrace{
    \frac{\partial J }{\partial  \nt{r}{s}{l}  }
    }_{\dd  \nt{r}{s}{l} } 
   =
   \underbrace{
    \frac{\partial J  }{\partial \nz{ij}{l} }
    }_{\dd  \nz{ij }{l} }\,\,\,\,  \times
    \underbrace{
    \frac{\partial \nz{ij}{l} }{\partial \nt{r}{s}{l}  }
    }_{ \na{(iS^{[l]}+r)(jS^{[l]}+s)}{l} }
    %\label{eq:}
\end{equation}

%\frac{ i-r}{S^{[l]}}\,  \frac{ j-s}{S^{[l]}} 

\begin{align}
\underbrace{
\frac{\partial J  }{\partial \nz{ij }{l}   }
}_{\dd  \nz{ij }{l} }
	&=
	\underbrace{ 
	\frac{\partial J  }{\partial \nz{uv}{l+1}   } 
	\frac{\partial \nz{uv}{l+1}   }{ \partial \na{ (uS^{[l]} +r )(vS^{[l]}+s)}{l}  }
	\delta_{i (uS^{[l]} +r )} \delta_{j (vS^{[l]} +s )}
	}_{\dd a_{ij}^{[l]} }
	\frac{\partial \na{ij}{l}    }{ \partial \nz{ij}{l}  }
	\\
	&=
	\underbrace{
	\frac{\partial J  }{\partial \nz{ uv  }{l+1}   } 
	 \nt{r}{s}{l+1}
	 \delta_{i (uS^{[l]} +r )} \delta_{j (vS^{[l]} +s )}
	 }_{\dd a_{ij}^{[l]}
	 =\theta^{[l+1]}_{ rs} \dd z_{uv}^{[l+1]} 
	  \delta_{i (uS^{[l]} +r )} \delta_{j (vS^{[l]} +s )}
	 } 
	 g^{[l]'}\left( z_{ij}   \right) 
\end{align}

Hence, in an abbreviated notation one has:
\begin{eqnarray}
	&\dd  \nt{j_l}{\mu}{l} = \dd  \nz{j_{l}}{l}  \na{j_{l-1}}{l-1}\\
	&\dd a_{j_{l-1}}^{[l-1]}=(\theta^{[l]T})_{j_{l-1} j_{l}} \dd z_{j_{l}}^{[l]}\\
	&\dd  \nz{j_{l}}{l}= \dd a_{j_l}^{[l]} \,\, g^{[l]'}\left( z_{j_{l}}^{[l]}   \right)
\end{eqnarray}

\section{Propagation with softmax }


\begin{equation}
    \na{\mu}{0} \equiv \delta_{\mu0}+\delta_{\mu i} x_i %\label{eq:}
\end{equation}

\begin{equation}
    \na{\mu}{l} \equiv \delta_{\mu0} +\delta_{\mu i }^{[l]}\,\,
    \ff{g^{[l]}}{  \nt{i}{\nu}{l}\na{\nu }{l-1}  } , \,\,\,\, l<L %\label{eq:} , 
\end{equation}

\begin{equation}
    \nz{i}{l}\equiv \nt{i}{\nu}{l}\na{\nu }{l-1} ,\,\,\,\, l\le L %\label{eq:} 
\end{equation}

\begin{align}
 \hat{y}_i \equiv   \frac{e^{\nz{i}{L}} } {\sum_c e^{\nz{c}{L} }}
\end{align}

It should be understood that in $\na{i}{l}$ 
one has  $i \in \{1,..., s_l\} $, 
and that in $\na{\mu}{l}$ one has $ \mu \in \{0,..., s_l\}$. The loss function is

\begin{equation}
    J\equiv -\sum_i y_i \ff{\log}{\hat{y}_i }  
\end{equation}

\section{Back propagation}

The following recursion relations do the job
\begin{equation}
    \underbrace{
    \frac{\partial J }{\partial  \nt{j_l}{j_{l-1}}{l}  }
    }_{\dd  \nt{j_l}{j_{l-1}}{l} } 
   =
   \underbrace{
    \frac{\partial J  }{\partial \nz{j_l}{l} }
    }_{\dd  \nz{j_{l}}{l} }
    \underbrace{
    \frac{\partial \nz{j_l}{l} }{\partial \nt{j_l}{j_{l-1}}{l}  }
    }_{ \na{j_{l-1}}{l-1} }
    %\label{eq:}
\end{equation}



\begin{equation}
\underbrace{
\frac{\partial J  }{\partial \nz{j_l}{l}   }
}_{\dd  \nz{j_{l}}{l} }
	=
	\underbrace{ 
	\frac{\partial J  }{\partial \nz{j_{l+1}}{l+1}   } 
	\frac{\partial \nz{j_{l+1}}{l+1}   }{ \partial \na{j_{l}}{l}  }
	}_{\dd a_{j_l}^{[l]} }
	\frac{\partial \na{j_{l}}{l}    }{ \partial \nz{j_{l}}{l}  }
	=
	\underbrace{
	\frac{\partial J  }{\partial \nz{j_{l+1}}{l+1}   } 
	 \nt{j_{l+1}}{j_l}{l+1}
	 }_{ \dd a_{j_l}^{[l]}=(\theta^{[l+1]T})_{j_l j_{l+1}} \dd z_{j_{l+1}}^{[l+1]} }
	 g^{[l]'}\left( z_{j_{l}}^{[l]}   \right)
\end{equation}

Hence, in an abbreviated notation one has:
\begin{eqnarray}
	&\dd  \nt{j_l}{j_{l-1}}{l} = \dd  \nz{j_{l}}{l}  \na{j_{l-1}}{l-1}\\
	&\dd a_{j_{l-1}}^{[l-1]}=(\theta^{[l]T})_{j_{l-1} j_{l}} \dd z_{j_{l}}^{[l]}\\
	&\dd  \nz{j_{l}}{l}= \dd a_{j_l}^{[l]} \,\, g^{[l]'}\left( z_{j_{l}}^{[l]}   \right)
\end{eqnarray}

Hence, by providing $\{ \dd a^{[l]}  , \theta^{[l]}\}$ one gets $\{ \dd  \nt{j_l}{\mu}{l}, \dd a_{j_{l-1}}^{[l-1]}\}$ and this process can be iterated. 
The initial condition is
\begin{eqnarray}
    &\dd \theta^{[L]}_{j_L j_{L-1}}= \frac{\partial J }{\partial  \nt{j_L}{\mu}{L}  } =
   \left( -  \frac{y_i }{ \hat{y}_i} \right)
   \left(  \frac{ \partial \hat{y}_i}{\partial z_{j_L}^{[L]} }    \right)
   \left(   \frac{ \partial z_{j_L}^{[L]} } {\partial  \nt{j_L}{j_{L-1}}{L}  }  \right) 
   \label{eq:softmaxp1}
\end{eqnarray}

\begin{align}
 \frac{ \partial \hat{y}_i}{\partial z_{j_L}^{[L]} } =\hat{y}_i(  \delta_{i j_L} -   \hat{y}_{j_L}) 
 \label{eq:softmaxp2}
\end{align}
\begin{align}
\frac{ \partial z_{j_L}^{[L]} } {\partial  \nt{j_L}{j_{L-1}}{L}  }  =  a^{L-1}_{j_{L-1}}
\label{eq:softmaxp3}
\end{align}
Note that after summing over $i$, the first two terms in \eqref{eq:softmaxp1} simplify as:
\begin{eqnarray}
   \left( -  \frac{y_i }{ \hat{y}_i} \right) \left(  \frac{ \partial \hat{y}_i}{\partial z_{j_L}^{[L]} }    \right) = - y_i + \hat{y}_i
\end{eqnarray}


*******************************************************

\section{Old, don't look!}




%%%%%%%%%%%%%%%%%%%%%%%%%%%%%%%%%%%%%%%%%%%

\section{Machine Learning coursera Andrew NG}

In the lectures notes, one has 
$\theta^{[l]}_{\text{here}}=\theta^{[l-1]}_{\text{there}}$.
We do this change to make the connexion with the 
deep learning specialization lectures.

\begin{equation}
    \na{\mu}{0} \equiv \delta_{\mu0}+\delta_{\mu i} x_i %\label{eq:}
\end{equation}

\begin{equation}
    \na{\mu}{l} \equiv \delta_{\mu0} +\delta_{\mu i }^{[l]}\,\,
    \ff{g^{[l]}}{  \nt{i}{\nu}{l}\na{\nu }{l-1}  }  %\label{eq:}
\end{equation}

\begin{equation}
    \nz{i}{l}\equiv \nt{i}{\nu}{l}\na{\nu }{l} %\label{eq:}
\end{equation}



It should be understood that in $\na{i}{l}$ 
one has  $i \in \{1,..., s_l\} $, 
and that in $\na{\mu}{l}$ one has $ \mu \in \{0,..., s_l\}$.


The cost fucntion has the structure

\begin{equation}
    cost=\frac1m \sum^m_n 
    J \Big\vert_{x=x_n,y=y_n}  %\label{eq:}
\end{equation}
where

\begin{equation}
    J\equiv -\sum^{s_L}_k
    \left(
        y_k\ff{\log}{\na{k}{L}}  %\label{eq:}
        +(1-y_k)\ff{\log}{1-\na{k}{L}} 
    \right)
\end{equation}

To implement gradient descent we need 
the next partial derivatives ($l\in\{1,...,L\} $)

\begin{equation}
    \frac{\partial J }{\partial  \nt{j_l}{\mu}{l}  } 
%    = 
%    \frac{\partial J  }{\partial \na{j_L}{L}  }
%    \left(
%    \frac{\partial \na{j_L}{L}  }{\partial \na{j_{L-1}}{L-1}  }
%    \cdots
%    \frac{\partial \na{j_{l+1}}{l+1}   }{\partial \na{j_{l}}{l}   }
%    \right)
%     \frac{\partial \na{j_{l}}{l}   }{\partial \nz{j_{l}}{l}   }
%    \frac{\partial \nz{j_{l}}{l} 
%    %\left(
%    %   \nt{j_{l}}{\nu}{l} \na{\nu}{l-1}
%    % \right)
%                }
%        {\partial \nt{i}{\mu}{l}  }\\
   =\frac{\partial J  }{\partial \na{j_L}{L}  }
    \left(
    \frac{\partial \na{j_L}{L}  }{\partial \na{j_{L-1}}{L-1}  }
    \cdots
    \frac{\partial \na{j_{l+1}}{l+1}   }{\partial \na{j_l}{l}   }
    \right)
    \frac{\partial \na{j_l}{l}   }{\partial \nz{j_l}{l}   }
    \frac{\partial \nz{j_l}{l} }
        {\partial \nt{j_l}{\mu}{l}  }
    %\label{eq:}
\end{equation}
through simple partial derivation one can show that 

\begin{equation}
    \frac{\partial J  }{\partial \na{j_L}{L}  }
    = \frac{(\na{j_L}{L} -y_{j_L})}{\na{j_L}{L} (1-\na{j_L}{L} )} 
    %a^{[L]'}_{j_L}  %\label{eq:}
\end{equation}


\begin{equation}
  \frac{\partial \nz{j_l}{l}  }
        {\partial \nt{j_l}{\mu}{l}  }
        =  \na{\mu}{l-1}
    %\label{eq:}
\end{equation}

\begin{equation}
    \frac{\partial \na{j_l}{l}  }
    {\partial \na{j_{l-1}}{l-1}  }
    =
   \left( \frac{\partial \na{j_l}{l}}   {\partial \nz{j_{l}}{l}  }\right)
   \left( \frac{\partial \nz{j_{l}}{l} }   {\partial \na{j_{l-1}}{l-1}  }\right)
    = \left( a^{[l]'}_{j_{l} }\right)
    \, \left( \nt{j_l}{j_{l-1}}{l} \right)
     %\label{eq:}
\end{equation}

\begin{equation}
	\frac{\partial \na{j_{l}}{l}   }{\partial \nz{j_{l}}{l}   } =  a_{j_{l}}^{[l]'}
\end{equation} 

Then, we can write
\begin{equation}
    \frac{\partial J }{\partial  \nt{j_l}{\mu}{l}  } =
    \frac{(\na{j_L}{L} -y_{j_L})}{\na{j_L}{L} (1-\na{j_L}{L-1} )} 
    \left(
    \frac{\partial \na{j_L}{L}  }{\partial \na{j_{L-1}}{L-1}  }
    \cdots\frac{\partial \na{j_{l+1}}{l+1}   }{\partial \na{j_{l}}{l}   } 
    \right) a_{j_{l}}^{[l]'} 
   \, \na{\mu}{l-1} 
    %\label{eq:}
\end{equation}

\section{Recursion relation}
Let us write
\begin{equation}
    \frac{\partial J }{\partial  \nt{j_l}{\mu}{l}  } 
   =\frac{\partial J  }{\partial \na{j_l}{l}   }
    \frac{\partial \na{j_l}{l}   }{\partial \nz{j_l}{l}   }
    \frac{\partial \nz{j_l}{l} }
        {\partial \nt{j_l}{\mu}{l}  }
    %\label{eq:}
\end{equation}
since 
\begin{equation}
\frac{\partial J  }{\partial \na{j_l}{l}   }
= \frac{\partial J  }{\partial \na{j_{l+1}}{l+1}   } \frac{\partial \na{j_{l+1}}{l+1}   }{ \partial \na{j_{l}}{l}  }
= \frac{\partial J  }{\partial \na{j_{l+1}}{l+1}   } 
  \left( \frac{\partial \na{j_{l+1}}{l+1}}   {\partial \nz{j_{l+1}}{l+1}  }\right)
   \left( \frac{\partial \nz{j_{l+1}}{l+1} }   {\partial \na{j_{l}}{l}  }\right)
=\frac{\partial J  }{\partial \na{j_{l+1}}{l+1}   } a^{[l+1]'}_{j_{l+1} }  \nt{j_{l+1}}{j_{l}}{l+1}
\end{equation}
In Andrew's program they use the following notation:
\begin{equation}
\underbrace{\frac{\partial J  }{\partial \na{j_l}{l}   }}_{\dd z_{j_l}^{[l]} }
=\underbrace{\nt{j_{l+1}}{j_{l}}{l+1} \frac{\partial J  }{\partial \na{j_{l+1}}{l+1}   }   }
_{ \dd a_{j_{l}}^{[l]}  =  (\theta^{[l+1]T} )_{j_{l} j_{l+1}}   \dd z_{j_{l+1}}^{[l+1]}  } 
a^{[l+1]'}_{j_{l+1} }
\end{equation}

%%%%%%%%%%%%%%%%%%%%%%%%%%%%%%%%%%%%%%%%%%%%%



\end{document}
